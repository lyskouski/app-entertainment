% Copyright 2023 The terCAD team. All rights reserved.
% Use of this content is governed by a CC BY-NC-ND 4.0 license that can be found in the LICENSE file.

\subsection{Kotlin}

Kotlin, a versatile and multiparadigm programming language, places a strong emphasis on safety, conciseness, and 
interoperability. Originally conceived to provide a more advanced alternative to Java, Kotlin has evolved far beyond 
its initial purpose. It has seamlessly expanded to cater diverse platforms, including the Java Virtual Machine (JVM), 
Android, JavaScript, and native applications. This evolution positions Kotlin as a dynamic language that not only 
prioritizes developer-friendly features but also effortlessly adapts to the demands of various programming environments.


\subsubsection{Supporting Multiparadigm}

At its inception, Kotlin's commitment to multiparadigmality went beyond the confines of traditional object-oriented 
programming, extending its support to functional programming as well. This dual-paradigm approach empowers developers 
with enhanced programming flexibility and the ability to seamlessly blend object-oriented and functional programming 
concepts. Developers can leverage the strengths of object-oriented programming for modeling and organizing code 
hierarchies, while simultaneously harnessing the power of functional programming for tasks such as handling collections, 
immutability, and higher-order functions.


\subsubsection{Unlocking Multiplatform}

While Kotlin continues to be extensively used for JVM and \q{Android} development, its adaptability has given rise to 
extensive multiplatform capabilities, reaching across various environments. The code compiles directly to native machine 
code, eliminating the need for a virtual machine. This native compilation makes Kotlin well-suited for \q{iOS} 
development, aligning with the typical use of native binaries in this ecosystem.

Kotlin supports the development of browser-based applications and Node.js applications using a unified language and 
codebase. Additionally, it facilitates the creation of \q{JavaScript} libraries, enabling seamless code sharing between 
Kotlin and JavaScript projects. Furthermore, Kotlin can be compiled to \q{WebAssembly}, allowing developers to harness 
its features for building web applications that achieve near-native performance in the browser.

In the realm of native applications, Kotlin offers support for developing applications and libraries on \q{macOS}, 
\q{Linux}, and \q{Windows} platforms.


\newpage
\subsubsection{Defining Safety}

Kotlin leverages a powerful type inference system, meaning that the compiler can automatically infer the type of a 
variable based on its initialization, reducing verbosity in the code.

\begin{lstlisting}
// Explicit type declaration
val name: String = "John"
// Type inference
val age = 25 // infers the type as Int
\end{lstlisting}

\noindent By default, variables in Kotlin are non-nullable, meaning they cannot hold a \q{null}-value. Nullable type 
should be explicitly marked with a nullable type modifier, denoted by the \q{?}-symbol:

\begin{lstlisting}
val nullableName: String? = null
val length: Int? = someNullableString?.length
\end{lstlisting}

\noindent When the compiler detects a certain type check, it automatically casts the variable to that type within the 
corresponding code block. This eliminates the need for explicit casting and enhances type safety:

\begin{lstlisting}
fun printLength(text: Any) {
  if (text is String) {
    // Within this block, 'text' is automatically cast to String
    println("Length: ${text.length}")
  }
}
\end{lstlisting}

\noindent Kotlin restricts programmers from defining custom operators, emphasizing clarity. Instead, it places a 
deliberate focus on readability and maintainability by allowing the redefinition of existing operators within specific 
contexts:

\begin{lstlisting}
// Custom class representing a point in 2D space
data class Point(val x: Int, val y: Int)

// Operator overloading for the plus operator (+)
operator fun Point.plus(other: Point): Point {
  return Point(this.x + other.x, this.y + other.y)
}

fun main() {
  // Creating instances of the Point class
  val point1 = Point(1, 2)
  val point2 = Point(3, 4)
  val sumPoint = point1 + point2

  // Displaying the result
  println("Sum of Points: $sumPoint")
}
\end{lstlisting}


\newpage
\subsubsection{Revising Conciseness Notation}

Conciseness refers to the language's ability to express functionality in a clear and compact manner, reducing 
unnecessary boilerplate code without sacrificing readability:

\begin{lstlisting}
// Using the Elvis operator
val result = nullableValue ?: "Default Value"

// Single-expression function
fun square(number: Int): Int = number * number

// Function with default parameter
fun greet(name: String, greeting: String = "Hello") {
  println("$greeting, $name!")
}

// Extension function to capitalize a string
fun String.capitalize(): String {
  return this.substring(0, 1).toUpperCase() + this.substring(1)
} // println("hello".capitalize()) // 'Hello'

val numbers = listOf(1, 2, 3, 4, 5) /* or */ val numbers = 1..5
// Using a lambda expression for filtering
val evenNumbers = numbers.filter { it % 2 == 0 }.sum()

// Data class definition
data class Person(val name: String, val age: Int)
val person1 = Person("John", 30)
val person2 = Person("John", 30)
// Automatically generated toString(), equals(), and hashCode()
println(person1) // 'Person(name=John, age=30)'
println(person1 == person2) // 'true'
// Destructuring declaration
val (name, age) = person1

// Lazy Initialization
val lazyValue: String by lazy {
  println("Initializing lazy value")
  "Lazy Initializer Result"
}

// Conditional statements
val result = when (score) {
  in 90..100 -> "A"
  in 80 until 90 -> "B"
  else -> "C"
}
\end{lstlisting}


\subsubsection{Following Interoperability}

Kotlin is designed to coexist within the broader Java ecosystem. Its interoperability is bidirectional, meaning that 
not only can Kotlin effortlessly leverage existing Java libraries and code, but Kotlin code itself can seamlessly 
integrate into Java projects. Kotlin supports Java annotations, and Kotlin's own annotations can be easily processed by 
Java tools. Finally, the Kotlin compiler generates bytecode that is fully compatible with the Java Virtual Machine (JVM), 
enabling Java projects to effortlessly incorporate Kotlin modules. 

\begin{lstlisting}
// src/main/java/com/example/JavaClass.java
package com.example

public class JavaClass {
  public void greetFromJava() {
    System.out.println("Hello from Java!");
  }(*@ \stopnumber @*)
} 

// src/main/kotlin/com/example/KotlinClass.kt 
package com.example

class KotlinClass {
  fun greetFromKotlin() {
    println("Hello from Kotlin!")
  }(*@ \stopnumber @*)
}

// src/main/kotlin/com/example/MixedUsageExample.kt
package com.example

fun main() {
  val javaClass = JavaClass()
  val kotlinClass = KotlinClass()

  javaClass.greetFromJava()
  kotlinClass.greetFromKotlin()
}
\end{lstlisting}

\noindent As Kotlin evolved and expanded its reach beyond the Java Virtual Machine (JVM) to include various platforms, 
the scope of interoperability guarantees was broadened. These guarantees now encompass not only interaction with Java 
code on the JVM but also extend to seamlessly integrate with JavaScript (JS) code for the JS-based platform. 
Additionally, Kotlin ensures robust interoperability with native applications written in languages such as C, C++, 
Objective-C, and Swift.


\subsubsection{Exploring Object-Oriented Programming}

Kotlin, as an object-oriented programming language \cite{Khan18}, embraces fundamental OOP principles. The available 
visibility modifiers are \q{public}, \q{protected} (visible within the class and its subclasses), \q{internal} (visible 
within the same module [a set of files compiled together]), and \q{private} (visible within the class in which they are 
declared). The visibility modifiers can be applied not only to methods but also to properties, classes, and other 
declarations. 

\begin{lstlisting}
// Encapsulation
class Person(private var name: String, private var age: Int) {
  fun getName(): String {
    return name
  }
  protected fun setName(newName: String) {
    name = newName
  }
}

// Abstraction
abstract class Shape {
  abstract fun area(): Double
}
// Concrete class extending the abstract class
class Circle(private val radius: Double) : Shape() {
  override fun area(): Double {
    return Math.PI * radius * radius
  }
}

// Inheritance
open class Animal(val name: String)
// Derived class inheriting from Animal
class Dog(name: String, val breed: String) : Animal(name)

// Polymorphism
open class Drawable {
  open fun draw() {
    println("Drawing a shape")
  }
}
// Derived classes with overridden draw methods
class DrawableCircle : Drawable() {
  override fun draw() {
    println("Drawing a circle")
  }
}
\end{lstlisting}


\subsubsection{Checking Functional Programming}

Functional Programming (FP) is a programming paradigm that treats computation as the evaluation of mathematical 
functions and avoids changing state and mutable data. Kotlin, being a modern programming language, supports functional 
programming \cite{Carl22} features:

\begin{lstlisting}
val numbers = 1..5
// Using map to square each element
val squaredNumbers = numbers.map { it * it }

// Modify an immutable data class results in a new instance
val person = Person(name = "John", age = 30)
val updatedPerson = person.copy(age = 31)

// Lambda expression
val add: (Int, Int) -> Int = { a, b -> a + b }
val result = add(3, 5)

// Higher-order function that takes a lambda as a parameter
fun performOperation(number: Int, operation: (Int) -> Int): Int {
  return operation(number)
}
fun main() {
  val result = performOperation(5) { it * it }
}

// Function composition using extension functions
fun square(x: Int) = x * x
fun double(x: Int) = x * 2
infix fun ((Int) -> Int).compose(g: (Int) -> Int): 
    (Int) -> Int = { x -> this(g(x)) }
fun main() {
  val squareAndDouble = ::square compose ::double
  val result = squareAndDouble(4)
  println("Result of composition: $result")
}

// Pattern matching
sealed class Result
data class Success(val value: Int) : Result()
object Failure : Result()
fun getResultMessage(result: Result): String {
    return when (result) {
        is Success -> "Success: ${result.value}"
        is Failure -> "Failure"
    }
}
\end{lstlisting}


\subsubsection{Observing Concurrency (Coroutines)}

Kotlin leverages the concept of suspendable computations, enabling the support for concurrency-related programming 
patterns like async/await, futures, promises, and actors. The Coroutines framework \cite{Babi22} offers a robust, 
elegant, and highly scalable solution for handling concurrency challenges:

\begin{lstlisting}
import kotlinx.coroutines.*

// Example of a coroutine
suspend fun doAsyncTask() { /* ... asynchronous logic */ }
// Flexible Thread Dispatching Mechanism
suspend fun performOperation() {
  withContext(Dispatchers.IO) { /* ... perform I/O operation */ }
}
// Suspendable Sequences and Iterators
suspend fun generateSequence(): Sequence<Int> = sequence {
  // ... yield values asynchronously
}
// Sharing Memory via Channels
fun channelCommunicationExample() = runBlocking {
  val channel = Channel<String>()
  launch { channel.send("Message from coroutine") }
  val receivedMessage = channel.receive()
  println("Received: $receivedMessage")
}
// Using Actors to Share Mutable State via Message Sending
class CounterActor {
  private var count = 0
  suspend fun increment() = count++
  suspend fun getCount(): Int = count
}

// Main coroutine context
fun main() = runBlocking {
    val task1 = async { doAsyncTask() }
    val task2 = async { performOperation() }
    val sequence = async { generateSequence().toList() }
    val task3 = async { channelCommunicationExample() }
    val counterActor = CounterActor()
    val task4 = async {
        counterActor.increment()
        println("Current Count: ${counterActor.getCount()}")
    }
    // Await for all tasks to complete
    awaitAll(task1, task2, task3, task4)
}
\end{lstlisting}
