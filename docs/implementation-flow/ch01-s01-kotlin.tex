% Copyright 2023 The terCAD team. All rights reserved.
% Use of this content is governed by a CC BY-NC-ND 4.0 license that can be found in the LICENSE file.

\subsection{Kotlin}

Kotlin, a versatile and multiparadigm programming language, places a strong emphasis on safety, conciseness, and 
interoperability. Originally conceived to provide a more advanced alternative to Java, Kotlin has evolved far beyond 
its initial purpose. It has seamlessly expanded its reach to cater to diverse platforms, including the Java Virtual 
Machine (JVM), Android, JavaScript, and native applications. This evolution positions Kotlin as a dynamic language that 
not only prioritizes developer-friendly features but also effortlessly adapts to the demands of various programming 
environments.


\subsubsection{Supporting Multiparadigm}

At its inception, Kotlin's commitment to multiparadigmality went beyond the confines of traditional object-oriented 
programming, extending its support to functional programming as well. This dual-paradigm approach empowers developers 
with enhanced programming flexibility and the ability to seamlessly blend object-oriented and functional programming 
concepts. Developers can leverage the strengths of object-oriented programming for modeling and organizing code 
hierarchies, while simultaneously harnessing the power of functional programming for tasks such as handling collections, 
immutability, and higher-order functions.

\subsubsection{Defining Safety}

Kotlin leverages a powerful type inference system, meaning that the compiler can automatically infer the type of a 
variable based on its initialization, reducing verbosity in the code.

\begin{lstlisting}
// Explicit type declaration
val name: String = "John"
// Type inference
val age = 25 // infers the type as Int
\end{lstlisting}

\noindent By default, variables in Kotlin are non-nullable, meaning they cannot hold a \q{null}-value. Nullable type 
should be explicitly marked with a nullable type modifier, denoted by the \q{?}-symbol:

\begin{lstlisting}
val nullableName: String? = null
\end{lstlisting}

\noindent When the compiler detects a certain type check, it automatically casts the variable to that type within the 
corresponding code block. This eliminates the need for explicit casting and enhances type safety:

\begin{lstlisting}
fun printLength(text: Any) {
  if (text is String) {
    // Within this block, 'text' is automatically cast to String
    println("Length: ${text.length}")
  }
}
\end{lstlisting}

\noindent Kotlin restricts programmers from defining custom operators, emphasizing clarity (and preventing potential 
troubles from the obfuscation of operator meanings). Instead, it places a deliberate focus on readability and 
maintainability by allowing the redefinition of existing operators within specific contexts.


\subsubsection{Following Conciseness}



\subsubsection{Enabling Interoperability}

