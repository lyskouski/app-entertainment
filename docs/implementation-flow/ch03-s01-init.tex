% Copyright 2023 The terCAD team. All rights reserved.
% Use of this content is governed by a CC BY-NC-ND 4.0 license that can be found in the LICENSE file.

\subsection{Instantiating the Project}

Kotlin Multiplatform (KMP) enables code sharing across various platforms, including Android, iOS, web, desktop, and 
others. It's supposed to be used IntelliJ IDEA (\href{https://www.jetbrains.com/idea/}{https://www.jetbrains.com/idea/}) 
or Android Studio (\href{https://developer.android.com/studio}{https://developer.android.com/studio}). Both of them 
support creation of a Kotlin Multiplatform App project. For Android Studio it would be needed to install Kotlin plugin
(\cref{img:i-plugin}):

\img{init/install-plugin}{Android Studio: Installing a Plugin for Kotlin}{img:i-plugin}

\noindent Creation the KMP project is followed by filling an application name and sub folders binding (\cref{img:i-config}).

\img{init/configure}{Android Studio: Creating Kotlin Multiplatform App}{img:i-config}

\noindent After the project is created, it comprises the following modules \issue{1}{}: shared, androidApp, iosApp. The 
\q{androidApp}-module is a Kotlin module designed to compile into an Android application, it depends on and utilizes the 
shared module as a standard Android library. The \q{iosApp}-module is an Xcode project configured to compile into an iOS 
application. It relies on and incorporates the shared module as an iOS framework. The shared module can function as 
either a conventional framework or as a CocoaPods dependency, depending on the distribution choice made earlier in the 
iOS framework distribution process. The \q{shared}-module is a Kotlin module containing logic shared between Android and 
iOS applications. This codebase is common to both platforms. It is organized into three distinct source sets: androidMain, 
commonMain, and iosMain. In the Gradle context, a source set is a logical grouping of files, each with its dependencies. 
In Kotlin Multiplatform, various source sets within the shared module can target different platforms.

The primary challenge with this approach lies in its omission of desktop and web components. This method seems to rely 
on a "know-how" approach, demanding substantial efforts to comprehend the intricate details of integrating these missing 
components. 

As an example, by registering those components at the end of \q{shared/build.gradle.kts}-file, an error appears:

\begin{lstlisting}[language=terminal]
> Configure project :shared
e: file:///.../shared/build.gradle.kts:36:30: Unresolved reference: outputKinds
w: Missing 'androidTarget()' Kotlin target in multiplatform project 'shared (:shared)'.
The Android Gradle plugin was applied without creating a corresponding 'android()' Kotlin Target:

  kotlin {
    androidTarget() // < -- please register this Android target
  }
\end{lstlisting}

\noindent The issue was relevant to the incorrect order of instructions in \q{shared/build.gradle.kts}. By moving 
\q{linuxX64} to the top, the initial error is replaced by:

\begin{lstlisting}[language=terminal]
> Task :shared:compileKotlinLinuxX64 FAILED

FAILURE: Build failed with an exception.

* What went wrong:
Execution failed for task ':shared:compileKotlinLinuxX64'.
\end{lstlisting}

\noindent Each system would required own folder with a registry definitions \cite{Resa22} (\cref{img:i-desktop}).

\img{init/enable-desktop}{Extend Kotlin Multiplatform App for Desktop}{img:i-desktop}

\noindent That approach is not ideal and requires the initial background in Kotlin and Gradle to understand made changes. 
Furthermore, Kotlin Multiplatform (KMP) offers Kotlin as the programming language but doesn't inherently provide a user 
interface (UI). When crafting a UI for Android within the KMP framework, we might either write it in native code. For 
iOS, the options include using UIKit or the modern SwiftUI framework with Swift. In the context of desktop applications, 
we might use Java Swing. Such a variability might slow down our immersion in cross-platform development with Kotlin.

\img{init/wizard}{Kotlin Multiplatform Web Wizard}{img:i-wizard}

\noindent The \q{Compose Multiplatform UI}-framework \cite{JetB23} extends the code-sharing capabilities of KMP beyond 
application logic. With this framework, we can implement the user interface once and seamlessly utilize it across all 
targeted platforms, including iOS, Android, desktop, and web. This approach enhances efficiency by centralizing UI 
development, ensuring a consistent and coherent user experience across diverse platforms without the need for redundant, 
platform-specific implementations. The project initialization involves the Kotlin Multiplatform web wizard
(\href{https://kmp.jetbrains.com}{https://kmp.jetbrains.com}, \cref{img:i-wizard}). Afterwards, we may build and run 
our sample project (\cref{img:i-webrun}):


\begin{lstlisting}[language=terminal]
$ gradlew build

Starting a Gradle Daemon (subsequent builds will be faster)
...
BUILD SUCCESSFUL in 22s
114 actionable tasks: 4 executed, 110 up-to-date

# Run web version (or, 'gradlew run' for desktop)
$ gradlew wasmJsBrowserRun

Type-safe project accessors is an incubating feature.
...
<===========--> 90% EXECUTING [2m 13s]
> :composeApp:wasmJsBrowserDevelopmentRun > webpack 5.82.0 compiled successfully in 1498 ms
\end{lstlisting}

\img{init/web-run}{WebAssembly (WASM) build run}{img:i-webrun}
